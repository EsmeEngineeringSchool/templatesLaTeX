\documentclass[11pt,a4paper,twoside]{article} %twocolumn
\usepackage[__LANG__]{babel}
\usepackage[babel=true,kerning=true]{microtype}
\usepackage[utf8]{inputenc}
\usepackage[T1]{fontenc}
\usepackage{lmodern}
\usepackage[top=2cm,bottom=2.2cm,right=2.2cm,left=2.2cm]{geometry}
\usepackage{graphicx}
\graphicspath{{./img/}}
\usepackage{examen}
\usepackage{multirow}
\usepackage{tabularx,booktabs}
\usepackage{siunitx}
\sisetup{inter-unit-product=\ensuremath{{}\cdot{}}}
\usepackage{schemabloc}
\usepackage{kinematik}
\usepackage{amsmath}
\usepackage{MnSymbol}
\usepackage{mathfmv}
\usepackage{wrapfig}
\usepackage{pgf,tikz}
\usepackage{pgfplots}
\pgfplotsset{compat=newest}
\usepackage{tikzfmv}
\usepackage{tkz-euclide}
\usepackage{tikz-3dplot}
\usepackage{pst-3dplot}
\usetikzlibrary{calc}
\usetikzlibrary{tikzmark}
\usetikzlibrary{quotes,angles}
\usetikzlibrary{arrows.meta,bending,automata,positioning}
\usetikzlibrary{shapes.geometric}       
\usetikzlibrary{babel} 
\usetikzlibrary{decorations.text}
\usetikzlibrary{matrix}
\usetikzlibrary{fill.image}
\usepgfplotslibrary{fillbetween}
\usetikzlibrary{intersections}
\newcommand{\tikzmarkk}[2]{
        \tikz[overlay,remember picture,baseline] 
        \node[anchor=base] (#1) {$#2$};
}
%%%%%%%%%%%%%%%%%%%%%%%%%%%%%%%%%%%%%%%%%%%%%%%%%%%%%%%%%%%%%%%%%%%%%%%%%%%%%%%%
% Paquets fmv
%%%%%%%%%%%%%%%%%%%%%%%%%%%%%%%%%%%%%%%%%%%%%%%%%%%%%%%%%%%%%%%%%%%%%%%%%%%%%%%%
\usepackage{fig-planes}
\usepackage{vecteurs}
\usepackage{poles}
\usepackage{bm}
%\usepackage{stanli}
\newcolumntype{M}[1]{>{\centering\arraybackslash}m{#1}}
%%%%%%%%%%%%%%%%%%%%%%%%%%%%%%%%%%%%%%%%%%%%%%%%%%%%%%%%%%%%%%%%%%%%%%%%%%%%%%%%
%%%%%%%%%%%%%%%%%%%%%%%%%%%%%%%%%%%%%%%%%%%%%%%%%%%%%%%%%%%%%%%%%%%%%%%%%%%%%%%%
%Modifier les variables suivantes
\annee{2024-2025}                  
\promo{IngéSUP}                         % ex. IngéSUP, IngéSPE, Ingé1
\module{\fr{Systèmes Techniques}{Technical Systems}}
\epreuve{\fr{Examen Final}{Final Exam}} % ex. MidTerm, FinalExam, Rattrapage
\titreEval{\fr{Systèmes Mécaniques}{Mechanical Systems}} % ex. Dynamique et Déformation
\dureeEval{2}                           % ex. 2 (en nombre d'heures)
\esme{true}                             % examen ESME true or false
\documentautorise{false}
\moyencalcul{false}
\dispositions{true}
\grille{__GRILLE__}                   % document réponse true or false
\corrige{__CORRIGE__}
%%%%%%%%%%%%%%%%%%%%%%%%%%%%%%%%%%%%%%%%%%%%%%%%%%%%%%%%%%%%%%%%%%%%%%%%%%%%%%%%
\begin{document}
\maketitle
\thispagestyle{fancy}
%%%%%%%%%%%%%%%%%%%%%%%%%%%%%%%%%%%%%%%%%%%%%%%%%%%%%%%%%%%%%%%%%%%%%%%%%%%%%%%%

%%%%%%%%%%%%%%%%%%%%%%%%%%%%%%%%%%%%%%%%%%%%%%%%%%%%%%%%%%%%%%%%%%%%%%%%%%%%%%%%
\exercice{
    \fr{Produits scalaire et vectoriel}
       {Dot and cross products } \textbf{(2pts)} 
}
%%%%%%%%%%%%%%%%%%%%%%%%%%%%%%%%%%%%%%%%%%%%%%%%%%%%%%%%%%%%%%%%%%%%%%%%%%%%%%%%
\fr{%
Soit un repère orthonormé direct $\rep{O}{1}$ orienté d'un angle $\theta$ par rapport à 
un second repère orthonormé direct $\rep{O}{0}$ tel que $\theta$ soit l'angle entre 
les vecteurs de bases $(\xx{0},\xx{1})$ et $(\yy{0},\yy{1})$ et avec les deux vecteurs confondus $\zz{0}=\zz{1}$.
}{%
Let a direct orthonormal frame $\rep{O}{1}$ be oriented at an angle $\theta$ relatively 
to a second direct orthonormal frame $\rep{O}{0}$ such that $\theta$ is the angle between
the basis vectors $(\xx{0},\xx{1})$ and $(\yy{0},\yy{1})$ and such that $\zz{0}=\zz{1}$.
}%
\begin{center}
    \begin{tikzpicture}
    \figplanes[n][theta][][1][][0][]{[x][y][z][O][]}
    \end{tikzpicture}
\end{center}
%%%%%%%%%%%%%%%%%%%%%%%%%%%%%%%%%%%%%%%%%%%%%%%%%%%%%%%%%%%%%%%%%%%%%%%%%%%%%%%%
\question{%
\fr
{Donner le résultat des trois produits scalaires suivants :
$\xx{0}\cdot\xx{1}$, $\xx{1}\cdot\yy{1}$ et $\zz{0}\cdot\zz{1}$}
{Give the result of the following three dot products:
$\xx{0}\cdot\xx{1}$, $\xx{1}\cdot\yy{1}$ and $\zz{0}\cdot\zz{1}$}
[0.5 pt]}
\reponse[3cm]{%
\[
    \xx{0}\cdot\xx{1} = \cos\theta 
\]
\[
    \xx{1}\cdot\yy{1} = 0 
\]
\[
    \zz{0}\cdot\zz{1} = 1
\]
}%
\question{%
\fr
{Donner le résultat des trois produits vectoriels suivants :
$\xx{0}\land\yy{0}$, $\xx{1}\land\yy{1}$ et $\zz{0}\land\zz{1}$
}
{Give the result of the following three cross products:
$\xx{0}\times\yy{1}$, $\xx{1}\times\yy{1}$ and $\zz{0}\times\zz{1}$
}
[0.5 pt]}
\reponse[3cm]
{%
\[
    \xx{0}\land\yy{0} = \zz{0}
\]
\[
    \xx{1}\land\yy{1} = \zz{1}
\]
\[
    \zz{0}\land\zz{1} = \vnull 
\]
}%
\fr
{On définit un vecteur position $\oa$ tel que $\bm{\oa=\xx{1}+2\yy{1}}$}
{We define a position vector $\oa$ such that $\bm{\oa=\xx{1}+2\yy{1}}$}

\question{%
\fr
{Donner le résultat des trois produits scalaires suivants :
$\oa\cdot\xx{0}$, $\oa\cdot\yy{0}$ et $\oa\cdot\xx{1}$}
{Give the result of the folowing three dot products: 
$\oa\cdot\xx{0}$, $\oa\cdot\yy{0}$ and $\oa\cdot\xx{1}$}
[0.5 pt]}
\reponse[3cm]
{
\[
    \oa\cdot\xx{0}= \cos\theta-2\sin\theta
\]
\[
    \oa\cdot\yy{0}= \sin\theta+2\cos\theta
\]
\[
    \oa\cdot\xx{1}= 1
\]
}%

\question{%
\fr
{Donner le résultat des trois produits vectoriels suivants :
$\oa\times\xx{0}$, $\oa\times\yy{0}$ et $\oa\times\xx{1}$}
{Give the result of the folowing three cross products: 
$\oa\times\xx{0}$, $\oa\times\yy{0}$ and $\oa\times\xx{1}$}
[0.5 pt]}
\reponse[3cm]
{
\[
    \oa\times\xx{0}=(-\sin\theta-2\cos\theta)\zz{0}
\]
\[
    \oa\times\yy{0}=-\sin\theta\zz{0}
\]
\[
    \oa\times\xx{1}=-2\zz{1}
\]
}[\clearpage]
%%%%%%%%%%%%%%%%%%%%%%%%%%%%%%%%%%%%%%%%%%%%%%%%%%%%%%%%%%%%%%%%%%%%%%%%%%%%%%%%
\exercice{
    \fr{Calcul vectoriel et centre de masse}
       {Vector calculus and center of mass} \textbf{(5pts)} 
}
%%%%%%%%%%%%%%%%%%%%%%%%%%%%%%%%%%%%%%%%%%%%%%%%%%%%%%%%%%%%%%%%%%%%%%%%%%%%%%%%
\fr{On étudie une plaque de bois (d'épaisseur négligeable) de masse surfacique  
homogène. La géométrie de masse sera donnée dans le repère associé $\rep{A}{1}$
comme représenté ci-dessous.
Rappel: la masse d'une plaque de surface $S$ de ce matériau sera donnée 
par $m=\sigma S$, où $\sigma$ est la masse surfacique.}
{We study a wooden plate (of negligible thickness) of homogeneous surface mass. 
The mass geometry will be given in the associated frame $\rep{A}{1}$ as shown below. 
Reminder: the mass of a plate of surface $S$ of this material will be given 
by $m=\sigma S$, where $\sigma$ is the surface mass.}

\definecolor{nnn}{rgb}{0.122,0.122,0.122}
\begin{center}
\begin{tikzpicture}
    \def\a{4}
    \def\ha{2.0}
    \def\hb{1.0}
    \coordinate (a) at (0   ,0);
    \coordinate (b) at (\a  ,0);
    \coordinate (c) at (\a  ,\a);
    \coordinate (d) at (\a-\hb ,\a);
    \coordinate (e) at (\a-\hb ,\ha);
    \coordinate (f) at (\hb   ,\ha);
    \coordinate (g) at (\hb   ,\a);
    \coordinate (h) at (0   ,\a);
    \draw[fill stretch image=img/Ikcfv.png] (a) -- 
                                            (b) -- 
                                            (c) -- 
                                            (d) -- 
                                            (e) -- 
                                            (f) -- 
                                            (g) -- 
                                            (h) -- cycle;
    \draw[decorate,decoration={brace,amplitude=2pt,mirror,raise=1ex}]  
         (a) -- (b) node[midway,yshift=-1.5em]{$a$};
    \draw[decorate,decoration={brace,amplitude=2pt,raise=1ex}]  
         (a) -- (h) node[midway,xshift=-1.5em]{$a$};
    \draw[decorate,decoration={brace,amplitude=2pt,raise=1ex}]  
         (d) -- (c) node[midway,yshift=1.5em]{$\frac{a}{4}$};
    \draw[decorate,decoration={brace,amplitude=2pt,raise=1ex}]  
         (h) -- (g) node[midway,yshift=1.5em]{$\frac{a}{4}$};
    \draw[decorate,decoration={brace,mirror,amplitude=2pt,raise=1ex}]  
         (d) -- (e) node[midway,xshift=-1.5em]{$\frac{a}{2}$};
    \draw[fill=nnn,draw=nnn] (a) circle[radius=1.5pt];
    \draw[fill=nnn,draw=nnn] (b) circle[radius=1.5pt];
    \draw[fill=nnn,draw=nnn] (c) circle[radius=1.5pt];
    \draw[fill=nnn,draw=nnn] (d) circle[radius=1.5pt];
    \draw[fill=nnn,draw=nnn] (e) circle[radius=1.5pt];
    \draw[fill=nnn,draw=nnn] (f) circle[radius=1.5pt];
    \draw[fill=nnn,draw=nnn] (g) circle[radius=1.5pt];
    \draw[fill=nnn,draw=nnn] (h) circle[radius=1.5pt];
    \node[below left] at (a) {A};
    \node[below right] at (b) {B};
    \node[above right] at (c) {C};
    \node[above left] at (d) {D};
    \node[below left] at (e) {E};
    \node[below right] at (f) {F};
    \node[above right] at (g) {H};
    \node[above left] at (h) {I};
    \draw[black,ultra thick, -latex] (a) --++ (5,0) node[below] {$\xx{1}$} ;
    \draw[black,ultra thick, -latex] (a) --++ (0,5) node[left] {$\yy{1}$} ;
\end{tikzpicture}
\end{center}

\question{
\fr
{Donner les vecteurs positions des points C,D et F  dans le repère $\rep{A}{1}$}
{Give the position vectors of points C, D and F in the frame $\rep{A}{1}$}
[1 pt]}
\reponse[5cm]
{%
\[
    \ac=a\xx{1}+a\yy{1}
\]
\[
    \ad=\dfrac{3a}{4}\xx{1}+a\yy{1}
\]
\[
    \af=\dfrac{a}{4}\xx{1}+\dfrac{a}{2}\yy{1}
\]
}%

\question{
\fr
{Donner la masse totale de la plaque en fonction des paramètres de la plaque, $\sigma$ et $a$}
{Give the total mass of the plate as a function of the plate parameters, $\sigma$ and $a$}
[1 pt]}
\reponse[6cm]
{
\[
    S=\dfrac{3}{4}a^2
\]
\[
    M=\sigma\dfrac{3}{4}a^2
\]
}%

\fr
{On rappel que le centre de masse $\ag$ d'un ensemble de solides $\SO[i]$ de masse $m_i$ et 
de centre de masse G$_i$ est donnée par : 
\[
    \ag=\dfrac{1}{M}\sum_i m_i\ag[][i].
\]
Par exemple un solide composé de deux plaques rectangulaires posséde un centre de masse donné par
le vecteur posision $\ag=\dfrac{1}{M}\left(m_1 \ag[][1] + m_2 \ag[][2]\right)$, où $M=m_1+m_2$.}
{We recall that the center of mass $\ag$ of a set of solids $\SO[i]$ of mass $m_i$ and center of 
mass G$_i$ is given by: 
\[
\ag=\dfrac{1}{M}\sum_i m_i\ag[][i]. 
\]
For example, a solid composed of two rectangular plates has a center of mass given by the position vector 
$\ag=\dfrac{1}{M}\left(m_1 \ag[][1] + m_2 \ag[][2]\right)$, where $M=m_1+m_2$.}

\question{
\fr
{Donner le centre de masse de la plaque, représenté par le vecteur position $\ag$, dans le repère $\rep{A}{1}$}
{Give the center of mass of the plate, represented by the position vector $\ag$, in the frame $\rep{A}{1}$}
[1 pt]}
\reponse[5cm]
{%
\[
    \ag=\dfrac{a}{2}\xx{1}+\dfrac{5a}{12}\yy{1}
\]
}[\clearpage]

\fr
{On considère maintenant la rotation de cette plaque autour de l'axe $\axe{A}{z}[0]$ d'un repère fixe $\rep{A}{0}$.
 On notera $\theta$, l'angle entre les vecteurs de bases $(\xx{0},\xx{1})$ et $(\yy{0},\yy{1})$.}
{We now consider the rotation of this plate around the axis $\axe{A}{z}[0]$ of a fixed frame $\rep{A}{0}$.
 We will denote $\theta$, the angle between the basis vectors $(\xx{0},\xx{1})$ and $(\yy{0},\yy{1})$.}
\question{
\fr
{Tracer la figure plane permettant de représenter cette rotation.}
{Draw the plane figure to represent this rotation.}
[1 pt]}
\reponse[5cm]
{
\begin{center}
    \begin{tikzpicture}
        \figplanes[][theta][][1][][0][]{[][][][A][][][]}
    \end{tikzpicture}
\end{center}
}

\question{
\fr
{À partir de cette représentation, donner le vecteur position $\ag$ dans le repère fixe $\rep{A}{0}$.} 
{From this representation, give the position vector $\ag$ in the fixed frame $\rep{A}{0}$.}
[1 pt]}
\reponse[6cm]
{
\[
    \xx{1}=\cos\theta\xx{0}+\sin\theta\yy{0}
\]

\[
    \yy{1}=-\sin\theta\xx{0}+\cos\theta\yy{0}
\]
\[
    \ag=\dfrac{a}{2}\left(\cos\theta\xx{0}+\sin\theta\yy{0}\right)+\dfrac{5a}{12}\left(-\sin\theta\xx{0}+\cos\theta\yy{0}\right)
\]
\[  
    \ag=\left(\dfrac{a}{2}\cos\theta-\dfrac{5a}{12}\sin\theta\right)\xx{0}+\left(\dfrac{a}{2}\sin\theta+\dfrac{5a}{12}\cos\theta\right)\yy{0}                                                                                                                             \]
}

\end{document}
