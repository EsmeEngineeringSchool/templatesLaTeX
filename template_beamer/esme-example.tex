\documentclass[aspectratio=169]{beamer}
\usepackage[french]{babel}
\usepackage[utf8]{inputenc}
\usepackage{fontenc}
\title{Modèle de diaporama ESME}
\subtitle{Sous titre}
\date[]{}
\author[]{}

\usetheme{esme}

\begin{document}
%%%%%%%%%%%%%%%%%%%%%%%%% TITLE PAGE %%%%%%%%%%%%%%%%%%%%%%%%%%%%%%%%%%%%%%%%%%%
\begin{frame}
\titlepage
\end{frame}
%%%%%%%%%%%%%%%%%%%%%%%%%%%%%%%%%%%%%%%%%%%%%%%%%%%%%%%%%%%%%%%%%%%%%%%%%%%%%%%%

%%%%%%%%%%%%%%%%%%%%%%%%%%%%%%%%%%%%%%%%%%%%%%%%%%%%%%%%%%%%%%%%%%%%%%%%%%%%%%%%
\section{Introduction}
\sectionframe{1}
\sectionframe{2}

%%%%%%%%%%%%%%%%%%%%%%%%%%%%%%%%%%%%%%%%%%%%%%%%%%%%%%%%%%%%%%%%%%%%%%%%%%%%%%%%

%%%%%%%%%%%%%%%%%%%%%%%%%%%%%%%%%%%%%%%%%%%%%%%%%%%%%%%%%%%%%%%%%%%%%%%%%%%%%%%%
\begin{frame} 
\frametitle{Introduction} 
\framesubtitle{Sous titre de la frame} 

Nous avons mis au point un thème \LaTeX~à destination des concepteurs de 
ressources pédagogiques numériques.\newline

Ce modèle prend la forme d'un \og template\fg~contenant des 
dispositions types.\newline

Ce modèle est conforme à la charte graphique de l'ESME (couleurs, logos).\newline

Les fonctionnalités de beamer sont conservées.

\end{frame}
%%%%%%%%%%%%%%%%%%%%%%%%%%%%%%%%%%%%%%%%%%%%%%%%%%%%%%%%%%%%%%%%%%%%%%%%%%%%%%%%

\begin{frame}{Table des Matières}
    \tableofcontents
    \scriptsize
    Utilisez les commandes \textbackslash section, \textbackslash subsection ...
    tout le long du document et obtenez cette page avec la commande 
    \textbackslash tableofcontents
\end{frame}

\section{Exemples}
\sectionframe{2}

\begin{frame}
    \frametitle{Exemple 1 d'itemize}
\begin{itemize}
\item one
\item two
\end{itemize}
\end{frame}
\end{document}
